\section{Описание структуры данных}

\begin{lstlisting}[caption={Структура для хранения длинного числа}, label={lst:listing1}]
	typedef struct 
	{
		char sign;     
		int mantissa[MANTISA_LEN];
		int exponent;         
		size_t man_length; 
	} bdouble_t;
\end{lstlisting}

\noindent\textbf{Объяснение полей:}
\newline
\begin{tabular}{|l|l|}
	\hline
	\textbf{Поле} & \textbf{Описание} \\
	\hline
	\texttt{sign} & Знак числа (+ или -) \\
	\hline
	\texttt{mantissa} & \makecell{Мантисса числа. Массив целых чисел,\\каждый элемент - одна цифра} \\
	\hline
	\texttt{exponent} & Порядок числа \\
	\hline
	\texttt{man\_length} & \makecell{Количество значимых\\элементов в мантиссе }\\
	\hline
\end{tabular}
