\section{Описание алгоритма}
\begin{enumerate}
	\item Ввод чисел
	\begin{itemize}
		\item Проверка корректности введенных чисел
		\item Запись чисел в структуру или вывод сообщения об ошибке
	\end{itemize}
	\item Деление целого числа на вещественное
	\begin{itemize}
		\item Создается переменная ans типа bdouble\_t, которая будет возвращена из функции. Происходит вычисление порядка: умножить на -1 экспоненту делителя и сложить с разностью экспоненты делимого и длины мантиссы делителя
		\item Начинается цикл с пост условием, пока не заполнено 40 цифр в мантиссе или делимое не равно 0
		\item Определяется длина неполного делимого и частное неполного делимого и делителя
		\item Находится вычитаемое для неполного делителя, за счет умножение делимого на частное 
		\item В мантиссу записывается частное неполного делимого и делителя 
		\item Из неполного делимого вычитается произведение делимого и частного
		\item Если уже заполнено 40 цифр в мантиссе или оставшееся число равно 0, то цикл деления завершается
	\end{itemize}
	\item Вывод результата деления
	\begin{itemize}
		\item Если делитель равен нулю, выводится сообщение о невозможности деления на ноль
		\item Если экспонента в результате больше 99999, выводится сообщение о достижении машинной бесконечности
		\item  Если экспонента в результате меньше 99999, выводится сообщение о достижении машинного нуля
		\item Иначе выводится ответ в формате [+-]?(0.[0-9]*)E[+-]?([0-9]*)
	\end{itemize}
\end{enumerate}

\textbf{Основные функции:}
\begin{enumerate}
	\item bdouble\_t (div\_big\_numbers(bdouble\_t *divident, bdouble\_t *divisor)
	\begin{itemize}
		\item Функция делит два длинных числа, представленных в виде структуры bdouble\_t. Возвращает результат деления, записанный в структуре bdouble\_t 
		\item divident: Делимое.
		\item divisor: Делитель.
	\end{itemize}
	
	\item size\_t incomplete\_dividend(size\_t *len, bdouble\_t *divident, const bdouble\_t *divisor)
	\begin{itemize}
		\item Функция определяет частное неполного делимого и делителя, и первым параметром изменяет длину неполного делимого.
		\item len: Длина неполного делимого.
		\item divident: Делимое.
		\item divisor: Делитель.
	\end{itemize}
	
	\item void mul\_big\_small(bdouble\_t *big, const int digit, const size\_t base)
	\begin{itemize}
		\item Функция умножает длинное число на цифру
		\item big: Длинное число
		\item digit: Цифра 
		\item base: Система счисления
	\end{itemize}
	
	\item int cmp\_mantissa(int *man1, const size\_t len1, int *man2, const size\_t len2)
	\begin{itemize}
		\item Функция сравнивает мантиссы двух длинных чисел
		\item man1: Мантисса первого числа
		\item len1: Длина мантиссы первого числа
		\item man2: Мантисса второго числа
		\item len2: Длина мантиссы второго числа
	\end{itemize}
	
	\item  int get\_sub(size\_t *len, bdouble\_t *subtrahend, bdouble\_t *divident, const bdouble\_t *divisor)
	\begin{itemize}
		\item Функция находит вычитаемое для неполного делимого. Возвращает очередную цифру частного делимого и делителя
		\item len: Длина неполного делимого
		\item \texttt{subtrahend}: Вычитаемое
		\item divident: Делимое
		\item divisor: Делитель
	\end{itemize}
\end{enumerate}
