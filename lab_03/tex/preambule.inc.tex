%----------------------- Преамбула -----------------------
\documentclass[utf8x, 14pt, oneside, a4paper]{extreport}

\usepackage{pgfplots}

\sloppy
\usepackage{tabularx}
\usepackage{booktabs}
\usepackage{array}
\usepackage{caption}
\usepackage{multirow}
\usepackage{setspace}
\usepackage{subcaption}
\usepackage{svg}
\usepackage{wrapfig}
\usepackage{float}
\usepackage{placeins} % в преамбуле

\usepackage{minted}
% Настройка для minted
\setminted{linenos, frame=single} % Рамка и фон для всех листингов
\captionsetup[listing]{labelformat=empty}

% Команда для создания подписи с "Листинг n"
\newcommand{\listingcaption}[1]{%
	\refstepcounter{figure} % Увеличиваем счётчик для правильной нумерации
	\caption{Листинг \thefigure: #1}%
}
\usepackage{extsizes} % Для добавления в параметры класса документа 14pt

\usepackage{makecell}
\usepackage{longtable}
\usepackage{wrapfig}
\usepackage{setspace}

\usepackage{algpseudocode}
\usepackage{algorithm}
\floatname{algorithm}{Алгоритм}

% Для работы с несколькими языками и шрифтом Times New Roman по-умолчанию
\usepackage[english,russian]{babel}
\usepackage{fontspec}
\setmainfont{Times New Roman}

% ГОСТовские настройки для полей и абзацев
\usepackage[left=30mm,right=10mm,top=20mm,bottom=20mm]{geometry}
\usepackage{misccorr}
\usepackage{indentfirst}
\usepackage{enumitem}
\setlist[enumerate,1]{label=\arabic*.} % Основной уровень
\setlist[enumerate,2]{label=\arabic{enumi}.\arabic*.} % Вложенный уровень
\setlist[enumerate,3]{label=\alph*)}

\setlength{\parindent}{1.25cm}
\renewcommand{\baselinestretch}{1.5}
\setlist{nolistsep} % Отсутствие отступов между элементами \enumerate и \itemize

% Дополнительное окружения для подписей
\usepackage{array}
\newenvironment{signstabular}[1][1]{
	\renewcommand*{\arraystretch}{#1}
	\tabular
}{
	\endtabular
}


\makeatletter
\renewcommand\LARGE{\@setfontsize\LARGE{22pt}{20}}
\renewcommand\Large{\@setfontsize\Large{20pt}{20}}
\renewcommand\large{\@setfontsize\large{16pt}{20}}
\makeatother
% Переопределение стандартных \chapter, \section, \subsection, \subsubsection по ГОСТу;
\RequirePackage{titlesec}
\titleformat{\chapter}[block]{\hspace{\parindent}\large\bfseries}{\thechapter}{0.5em}{\large\bfseries\raggedright}
\titleformat{name=\chapter,numberless}[block]{\hspace{\parindent}}{}{0pt}{\large\bfseries\centering}
\titleformat{\section}[block]{\hspace{\parindent}\large\bfseries}{\thesection}{0.5em}{\large\bfseries\raggedright}
\titleformat{\subsection}[block]{\hspace{\parindent}\large\bfseries}{\thesubsection}{0.5em}{\large\bfseries\raggedright}
\titleformat{\subsubsection}[block]{\hspace{\parindent}\large\bfseries}{\thesubsection}{0.5em}{\large\bfseries\raggedright}
\titlespacing{\chapter}{12.5mm}{-22pt}{10pt}
\titlespacing{\section}{12.5mm}{10pt}{10pt}
\titlespacing{\subsection}{12.5mm}{10pt}{10pt}
\titlespacing{\subsubsection}{12.5mm}{10pt}{10pt}

% Работа с изображениями и таблицами; переопределение названий по ГОСТу
\usepackage{caption}
\captionsetup[figure]{name={Рисунок},labelsep=endash}
\captionsetup[table]{singlelinecheck=false, labelsep=endash}

\usepackage{graphicx}
\usepackage{diagbox} % Диагональное разделение первой ячейки в таблицах

% Цвета для гиперссылок и листингов
\usepackage{color}

% Гиперссылки \toc с кликабельностью
\usepackage{hyperref}

\hypersetup{
	linktoc=all,
	linkcolor=black,
	colorlinks=true,
}

% Листинги
\setsansfont{Arial}
\setmonofont{Courier New}

\usepackage{color} % Цвета для гиперссылок и листингов
\definecolor{comment}{rgb}{0,0.5,0}
\definecolor{plain}{rgb}{0.2,0.2,0.2}
\definecolor{string}{rgb}{0.91,0.45,0.32}
\hypersetup{citecolor=blue}
\hypersetup{citecolor=black}

\DeclareCaptionLabelSeparator{line}{\ --\ }
\DeclareCaptionFont{white}{\color{white}}

\usepackage{ulem} % Нормальное нижнее подчеркивание
\usepackage{hhline} % Двойная горизонтальная линия в таблицах
\usepackage[,table]{totalcount} % Подсчет изображений, таблиц
\usepackage{rotating} % Поворот изображения вместе с названием
\usepackage{lastpage} % Для подсчета числа страниц

%\captionsetup[figure]{justification=centering,labelsep=dash}
%\captionsetup[table]{labelsep=dash,justification=raggedright,singlelinecheck=off}

\makeatletter
\renewcommand\@biblabel[1]{#1.}
\makeatother

\onehalfspacing

\pgfplotsset{width=0.85\linewidth, height=0.5\columnwidth}

\linespread{1.3}

\parindent=1.25cm

%\LetLtxMacro\itemold\item
%\renewcommand{\item}{\itemindent0.75cm\itemold}

\def\labelitemi{---}
\setlist[itemize]{leftmargin=1.25cm, itemindent=0.65cm}
\setlist[enumerate]{leftmargin=1.25cm, itemindent=0.55cm}


\usepackage{tabularx}
\usepackage{amsmath}
\usepackage{slashbox}